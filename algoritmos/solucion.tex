\documentclass[hidelinks,a4paper,10pt, nofootinbib]{article}
\usepackage[width=15.5cm, left=3cm, top=2.5cm, right=2cm, left=2cm, height= 24.5cm]{geometry}
\usepackage[spanish]{babel}
\usepackage[utf8]{inputenc}
\usepackage[T1]{fontenc}
\usepackage{amsmath}
\usepackage{amsfonts}
\usepackage{amssymb}
\usepackage[colorlinks=true, pagebackref=true]{hyperref}
\usepackage{xcolor}
\usepackage{enumerate}
\usepackage{mathpazo}

\setlength{\parindent}{4em}
\setlength{\parskip}{1em}

\title{Prueba de Oposición - Área Algoritmos}  
\author{Gonzalo Ciruelos}
\date{14 de septiembre de 2015} 

\newcommand{\tx}[1]{\text{#1}}
\newcommand{\secu}{\ensuremath\texttt{secu}(\alpha)}
\newcommand{\rev}{\text{Reverso}}
\newcommand{\cons}{\bullet}
\newcommand{\snoc}{\circ}
\newcommand{\nil}{\text{nil}}
\newcommand{\prim}{\text{prim}}
\newcommand{\fin}{\text{fin}}

\newcommand{\por}[1]{\overset{\text{#1}}{\equiv}}
\newcommand{\ph}{\phantom{rev1}}

\begin{document}
\maketitle

\section*{Ejercicio}

Demuestre por inducción estructural que:

$$(\forall s : \texttt{secu}(\alpha))(\tx{Reverso}(\tx{Reverso}(s)) =_{obs} s)$$

Plantee claramente los lemas necesarios y demostrarlos.


\section*{Solución}

Primero recordemos la definición de \rev.
\[\rev(\nil) \equiv \nil \tag{rev1}\]
\[\rev(a \cons s) \equiv Reverso(s) \snoc a \tag{rev2}\]

Además,
\[s \circ e \equiv \textbf{ if } \text{vacia?}(s) \textbf{ then } e \cons \nil \textbf{ else } \prim(s) \cons (\fin(s) \circ e) \tag{snoc}\]

Recordemos el esquema de inducción del TAD \texttt{secu}.

$$P(\nil)\  \land \  (\forall a : \alpha) (\forall s : \secu) (P(s) \implies P(a \cons s))$$

Si probamos esto, probamos que $(\forall s : \secu) P(s)$.

En este caso, $P$  (el \emph{predicado unario}) es
$$ P(s) \equiv \rev(\rev(s)) = s$$



\subsection*{Caso base}
El caso base en este caso es $P(\nil)$

\begin{center}
\begin{tabular}{c l}
$P(\nil)$ & $\por{\phantom{rev1}} \rev(\rev(\nil)) = \nil$ \\
          & $\por{rev1} \rev(\nil) = \nil$ \\
          & $\por{rev1} \nil = \nil$  \\
          & $\por{\phantom{rev1}} true$ \\
\end{tabular}
\end{center}

\subsection*{Caso inductivo}
El caso inductivo en este caso es $(\forall a : \alpha) (\forall s : \secu) (P(s) \implies P(a \cons s))$

Sean $a : \alpha$ y $s : \secu$. Supongo $P(s)$ y quiero llegar a $P(a \cons s)$

\begin{center}
\begin{tabular}{c l}
$P(a \cons s)$ & $\por{\ph} \rev(\rev(a \cons s)) = a \cons s$ \\
               & $\por{rev2} \rev(\rev(s) \circ a) = a \cons s$ \\
               & $\por{lema} a \cons \rev(\rev(s)) = a \cons s$ \\
               & $\por{ HI } a \cons s = a \cons s$ \\ 
               & $\por{\ph} true$ \\
\end{tabular}
\end{center}

\subsection*{Lema}

\[(\forall s : \secu) (\forall a : \alpha) (\rev(s \circ a) = a \cons \rev(s))\]

Para probarlo vamos a usar también inducción estructural, con 
\[P(s) \equiv (\forall a : \alpha) (\rev(s \circ a) = a \cons \rev(s))\]

\subsubsection*{Caso base}
$P(\nil) \equiv (\forall a : \alpha) (\rev(\nil \circ a) = a \cons \rev(\nil))$

Sea $ a : \alpha$, veamos que $  \rev(\nil \circ a) = a \cons \rev(\nil)$.

\begin{center}
\begin{tabular}{c l}
$\rev(\nil \circ a) = a \cons \rev(\nil)$ & $\por{snoc} \rev(a \cons \nil) = a \cons \rev(\nil)$ \\
               & $\por{rev2} \rev(\nil) \snoc a = a \cons \rev(\nil)$ \\
               & $\por{rev1} \nil \snoc a = a \cons \rev(\nil)$ \\
               & $\por{snoc} a \cons \nil = a \cons \rev(\nil)$ \\ 
               & $\por{rev1} a \cons \nil = a \cons \nil$ \\ 
               & $\por{\ph} true$ \\
\end{tabular}
\end{center}

\subsubsection*{Caso inductivo}
Queremos ver que \[(\forall s : \secu) (\forall e : \alpha) (P(s) \implies P(e \cons s)) \]


Sean $s : \secu$ y $e : \alpha$, y supongamos $P(s)$, es decir, \[(\forall a : \alpha) (\rev(s \circ a) = a \cons \rev(s))\].


Veamos entonces que sucede con \[P(e \cons s) \equiv (\forall a : \alpha) (\rev((e \cons s) \circ a) = a \cons \rev(e \cons s))\]

Sea $ a : \alpha$, 
\begin{center}
\begin{tabular}{c l}
$\rev((e \cons s) \snoc a) = a \cons \rev(e \cons s)$ & $\por{\ph} \rev(e \cons (s \snoc a)) = a \cons \rev(e \cons s)$ \\
               & $\por{rev2} \rev(s \snoc a) \snoc e= a \cons \rev(e \cons s)$ \\
               & $\por{ HI } (a \cons \rev(s)) \snoc e= a \cons \rev(e \cons s)$ \\
               & $\por{snoc} a \cons (\rev(s) \snoc e) = a \cons \rev(e \cons s)$ \\
               & $\por{rev2} a \cons (\rev(s) \snoc e) = a \cons (\rev(s) \snoc e)$ \\
               & $\por{\ph} true$ \\
\end{tabular}
\end{center}


Esto completa la demostración.

\end{document}

