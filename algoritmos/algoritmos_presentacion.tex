\documentclass{beamer}
\usepackage{beamerthemeshadow}
\usepackage[spanish]{babel}
\usepackage[utf8]{inputenc}
\usepackage[T1]{fontenc}
\usepackage{enumerate}
\usepackage{algpseudocode}

\newcommand{\tx}[1]{\text{#1}}
\newcommand{\secu}{\ensuremath\texttt{secu}(\alpha)}
\newcommand{\rev}{\text{Reverso}}
\newcommand{\cons}{\bullet}
\newcommand{\snoc}{\circ}
\newcommand{\nil}{\text{nil}}
\newcommand{\prim}{\text{prim}}
\newcommand{\fin}{\text{fin}}

\newcommand{\por}[1]{\overset{\text{#1}}{\equiv}}
\newcommand{\ph}{\phantom{rev1}}

\begin{document}
\title{Prueba de Oposición - Área Algoritmos}  
\author{Gonzalo Ciruelos}
\date{2 de noviembre de 2016}

\frame{\titlepage} 

\section{Presentación} 
\frame{\frametitle{Presentación} 
\begin{itemize}
    \item Materia :  \emph{Algoritmos y Estructuras de Datos II}
    \vspace{2em}
    \item Práctica : \emph{S\'eptima práctica - Dividir y Conquistar}
\end{itemize}
}

\section{Prácticas de la materia} 
\frame{\frametitle{Presentación} 
\begin{itemize}
    \item Primera parte
    \begin{itemize}
        \item Especificación con Tipos Abstractos de Datos
        \item Demostración de propiedades
        \item Diseño: invariante de representación y función de abstracción
    \end{itemize}
    \item Segunda Parte
    \begin{itemize}
        \item Complejidad Algorítmica
        \item Diseño: elección de estructuras de datos 
        \item Ordenamiento
        \item \textbf{Dividir y Conquistar}
    \end{itemize}
\end{itemize}
}

\section{Ejercicio} 
\frame{\frametitle{Contexto}
\begin{itemize}
    \item El ejercicio podría formar parte de una práctica o de una clase introductoria al tema.  
   \item Es bueno como introducción al tema porque para resolverlo es muy fácil pero permite ver el potencial de la t\'ecnica.
   \item Supone que los alumnos asistieron a la teórica del tema y que conocen la existencia del teorema maestro.
\end{itemize} 
}

\frame{\frametitle{Enunciado}
Encuentre un algoritmo para calcular $a^b$ en tiempo logarítmico en $b$.
Piense cómo reutilizar los resultados ya calculados.
Justifique la complejidad del algoritmo dado.
}

\frame{\frametitle{Idea de la solución}
  Notar que
  \[
    a^b = 
      \left\{
        \begin{array}{ll}
           1                              & \mbox{si $b = 0$}      \\
           a^{\frac{b}2} a^{\frac{b}2}    & \mbox{si $b > 0$ y $b$ es par}   \\
           a^{\frac{b-1}2} a^{\frac{b-1}2}  a & \mbox{si $b$ es impar} 
        \end{array}
      \right.
  \] 
}

\frame{\frametitle{Algoritmo}
\begin{algorithmic}
\Function{Pow}{$a,b$}
  \If {$b == 0\geq maxval$}
    \State \Return $1$
  \ElsIf {esPar($b$)}
    \State \Return Pow$\big(a, \frac{b}{2}\big)^2$
  \Else
    \State \Return Pow$\big(a, \frac{b-1}{2}\big)^2$ $\cdot$ $a$
  \EndIf
\EndFunction
\end{algorithmic}
}

\frame{\frametitle{Complejidad}
Formulación recursiva de la complejidad:

\[
 T(a, 0) = O(1)
\]

\[
  T(a,b) = T\bigg(a, \frac{b}2\bigg) + O(1)
\]

Como se ve, solo depende de $b$:
\[
 T(0) = O(1)
\]

\[
  T(b) = T\bigg(\frac{b}2\bigg) + O(1)
\]
}

\frame{\frametitle{Complejidad (cont.)}
Recordando el teorema maestro\dots

Supongamos que \[T(n) = a T\bigg(\frac{n}{b}\bigg) + f(n) \]
Entonces,
\begin{enumerate}
  \item Si $f(n) \in O(n^c)$ donde $c < \log_b a$, luego $T(n) \in \Theta(n^{\log_b a})$
  \item Si $f(n) \in O(n^{\log_b a})$, luego $T(n) \in \Theta(n^{\log_b a} \log n)$
  \item Si $f(n) \in O(n^c)$ donde $c > \log_b a$ y $a f(\frac{n}{b}) \leq k f(n)$ para alguna constante $k < 1$ y $n$ suficientemente grande, luego $T(n) \in \Theta(f(n))$
\end{enumerate}

}

\frame{\frametitle{Complejidad (cont.)}
En este caso,

\[ a = 1 \]
\[ b = 2 \]
\[ f(n) \in O(1) \]

Entonces, $f(n) \in O(1) = O(n^0)$ y $0 = log_2(1)$, entonces estamos en el caso \textbf{2}.

Luego, utilizando el Teorema Maestro, obtenemos que
\[T(n) \in \Theta(n^{log_2 1} \log n) = \Theta(\log n) \]
}


\frame{
\huge
\begin{center}
¡Gracias!

\vspace{1em}
¿Preguntas?
\end{center}
}


\end{document}
      

