\documentclass{beamer}
\usepackage{beamerthemeshadow}
\usepackage[spanish]{babel}
\usepackage[utf8]{inputenc}
\usepackage[T1]{fontenc}
\usepackage{enumerate}

\newcommand{\tx}[1]{\text{#1}}
\newcommand{\secu}{\ensuremath\texttt{secu}(\alpha)}
\newcommand{\rev}{\text{Reverso}}
\newcommand{\cons}{\bullet}
\newcommand{\snoc}{\circ}
\newcommand{\nil}{\text{nil}}
\newcommand{\prim}{\text{prim}}
\newcommand{\fin}{\text{fin}}

\newcommand{\por}[1]{\overset{\text{#1}}{\equiv}}
\newcommand{\ph}{\phantom{rev1}}

\begin{document}
\title{Prueba de Oposición - Área Algoritmos}  
\author{Gonzalo Ciruelos}
\date{14 de septiembre de 2015} 

\frame{\titlepage} 

\section{Presentación} 
\frame{\frametitle{Presentación} 
\begin{itemize}
    \item Materia :  \emph{Algoritmos y Estructuras de Datos II}
    \vspace{2em}
    \item Práctica : \emph{Segunda práctica - Demostración de propiedades}
\end{itemize}
}

\section{Prácticas de la materia} 
\frame{\frametitle{Presentación} 
\begin{itemize}
    \item Primera parte
    \begin{itemize}
        \item Especificación con Tipos Abstractos de Datos
        \item \textbf{Demostración de propiedades}
        \item Diseño: invariante de representación y función de abstracción
    \end{itemize}
    \item Segunda Parte
    \begin{itemize}
        \item Complejidad Algorítmica
        \item Diseño: elección de estructuras de datos 
        \item Ordenamiento
        \item Dividir y Conquistar
    \end{itemize}
\end{itemize}
}

\section{Ejercicio} 
\frame{\frametitle{Contexto}
\begin{itemize}
    \item El ejercicio podría formar parte de una práctica o de una clase introductoria al tema.  
   \item Es bueno como introducción al tema porque para resolverlo se utilizan técnicas comunes en todos los problemas de inducción estructural.
   \item Los alumnos terminaron la práctica de TADs, asistieron a la teórica del tema y empezaron a ejercitarlos.
\end{itemize} 
}

\frame{\frametitle{Enunciado}
Demuestre por inducción estructural que:

$$(\forall s : \texttt{secu}(\alpha))(\tx{Reverso}(\tx{Reverso}(s)) =_{obs} s)$$

Plantee claramente los lemas necesarios y demustr\'elos.
}

\frame{
Recordemos algunas definiciones, $(\forall a: \alpha) (\forall s : \secu$)
\[\rev(\nil) \equiv \nil \tag{rev1}\]
\[\rev(a \cons s) \equiv Reverso(s) \snoc a \tag{rev2}\]

Además,
\[s \circ e \equiv \textbf{ if } \text{vacia?}(s) \textbf{ then } e \cons \nil \textbf{ else } \prim(s) \cons (\fin(s) \circ e) \tag{snoc}\]

\[\prim(a \cons s) \equiv a  \tag{prim}\]
\[\fin(a \cons s) \equiv s  \tag{fin}\]
}


\frame{\frametitle{Solución}
Recordemos el esquema de inducción del TAD $\secu$.

$$P(\nil) \land ((\forall s : \secu)\ (P(s) \ \implies \ (\forall a : \alpha) P(a \cons s))$$

Si probamos esto, probamos que $(\forall s : \secu) P(s)$.
\pause

En este caso, $P$  (el \emph{predicado unario}) es
$$ P(s) \equiv \rev(\rev(s)) = s$$
}

\frame{\frametitle{Lema}
\[(\forall s : \secu) (\forall a : \alpha) (\rev(s \circ a) = a \cons \rev(s))\]

Para probarlo vamos a usar también inducción estructural, con 
\[P(s) \equiv (\forall a : \alpha) (\rev(s \circ a) = a \cons \rev(s))\]

}

\frame{\frametitle{Lema: Caso base}

$P(\nil) \equiv (\forall a : \alpha) (\rev(\nil \circ a) = a \cons \rev(\nil))$

Sea $ a : \alpha$, veamos que $  \rev(\nil \circ a) = a \cons \rev(\nil)$.

\begin{center}
\begin{tabular}{c l}
\ph            & $\por{snoc} \rev(a \cons \nil) = a \cons \rev(\nil)$ \\
\pause
               & $\por{rev2} \rev(\nil) \snoc a = a \cons \rev(\nil)$ \\
\pause
               & $\por{rev1} \nil \snoc a = a \cons \rev(\nil)$ \\
\pause
               & $\por{snoc} a \cons \nil = a \cons \rev(\nil)$ \\ 
\pause
               & $\por{rev1} a \cons \nil = a \cons \nil$ \\ 
\pause
               & $\por{\ph} true$ \\
\end{tabular}
\end{center}
}


\frame{\frametitle{Lema: Caso inductivo}
Queremos ver que \[(\forall s : \secu))(P(s) \implies (\forall e : \alpha) P(e \cons s)) \]


Sea $s : \secu$ y supongamos $P(s)$, es decir, \[(\forall a : \alpha) (\rev(s \circ a) = a \cons \rev(s))\]
\vspace{1em}

Sea $e : \alpha$. Veamos entonces que sucede con \[P(e \cons s) \equiv (\forall a : \alpha) (\rev((e \cons s) \circ a) = a \cons \rev(e \cons s))\]

}

\frame{\frametitle{Lema: Caso inductivo (cont.)}
Sea $ a : \alpha$, 
$\rev((e \cons s) \snoc a) = a \cons \rev(e \cons s)$  \\
   
\begin{center}
\begin{tabular}{r l}
\ph            & $\por{snoc} \rev(e \cons (s \snoc a)) = a \cons \rev(e \cons s)$ \\
\pause
               & $\por{rev2} \rev(s \snoc a) \snoc e= a \cons \rev(e \cons s)$ \\
\pause
               & $\por{ HI } (a \cons \rev(s)) \snoc e= a \cons \rev(e \cons s)$ \\
\pause
               & $\por{snoc} a \cons (\rev(s) \snoc e) = a \cons \rev(e \cons s)$ \\
\pause
               & $\por{rev2} a \cons (\rev(s) \snoc e) = a \cons (\rev(s) \snoc e)$ \\
\pause 
               & $\por{\ph} true$ \\
\end{tabular}
\end{center}
}

\frame{
\huge
\begin{center}
¡Gracias!

\vspace{1em}
¿Preguntas?
\end{center}
}


\end{document}
      

