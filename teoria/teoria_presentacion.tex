\documentclass{beamer}
\usepackage{beamerthemeshadow}
\usepackage[spanish]{babel}
\usepackage[utf8]{inputenc}
\usepackage[T1]{fontenc}
\usepackage{enumerate}

\newcommand{\ele}{\mathcal{L}}

\begin{document}
\title{Prueba de Oposición - Área Teoría}  
\author{Gonzalo Ciruelos}
\date{19 de septiembre de 2017} 

\frame{\titlepage} 

\section{Presentación} 
\frame{\frametitle{Presentación} 
\begin{itemize}
    \item Materia :  \emph{Teoría de Lenguajes}
    \vspace{2em}
    \item Práctica : \emph{Lenguajes regulares y lema de pumping}
\end{itemize}
}

\section{Prácticas de la materia} 
\frame{\frametitle{Presentación} 
\begin{itemize}
    \item Primera parte
    \begin{itemize}
		\item{Autómatas finitos y gramáticas regulares}
		\item{Expresiones regulares}
		\item{Autómatas finitos (cont.)}
		\item{\textbf{Lenguajes regulares y lema de pumping}}
		\item{Lenguajes libres de contexto}
		\item{Traductores finitos}
    \end{itemize}
    \item Segunda Parte
    \begin{itemize}
		\item{Escritura de gramáticas libres de contexto}
		\item{Parsers descendentes}
		\item{Gramáticas y parsers LR}
		\item{Gramáticas de atributos y TDS}
    \end{itemize}
\end{itemize}
}

\section{Ejercicio} 
\frame{\frametitle{Contexto}
\begin{itemize}
  \item El ejercicio podría ser un repaso de los temas de lema de pumping.
  \item El lema de pumping da una condición necesaria para que un lenguaje sea regular, pero no suficiente.
  \item Se asume que el alumno conoce el lema de pumping y además conoce propiedades de lenguajes regulares.
\end{itemize} 
}

\frame{\frametitle{Enunciado}

Dado $\ele = \{0^i 1^j\ |\ i > j \, \lor \, i \text{ es par}\}$

\begin{enumerate}[a.]
\item Demostrar que $\ele$ cumple
  \[\forall \alpha( \alpha \in \ele \land |\alpha| \geq 2 \implies \]
  \[
            \exists x \exists y  \exists z (
                        \alpha = x y z \land
                        |x y| \leq 2 \land
                        |y| \geq 1 \land
                        \forall i (x y^i z \in \ele))) \]
\item Demostrar que $\ele$ no es regular.
\end{enumerate}
}

\frame{\frametitle{Solución: cosas que usamos}

\begin{itemize}
\item Un lenguaje es regular si y solo si su complemento es regular.
\item Lema de \emph{pumping}: Si un lenguaje $\ele'$ es regular, entonces existe $n$ tal que todo $\gamma \in \ele'$ con $|\gamma| \geq n$ puede ser escrito como $\gamma = xyz$ con $x$, $y$, $z$ satisfaciendo
   \[|xy| \leq n\]
   \[|y| \geq 1 \]
   \[xy^rz \in \ele' \text{\ para todo $r\in\mathbb{N}_0$}\]
\end{itemize}
}

\frame{
\huge
\begin{center}
¿Preguntas?
\end{center}
}

\end{document}
      

