\documentclass{beamer}
\usepackage{beamerthemeshadow}
\usepackage[spanish]{babel}
\usepackage[utf8]{inputenc}
\usepackage[T1]{fontenc}
\usepackage{enumerate}

\newcommand{\ele}{\mathcal{L}}

\begin{document}
\title{Prueba de Oposición - Área Teoría}  
\author{Gonzalo Ciruelos}
\date{19 de septiembre de 2017} 

\frame{\titlepage} 

\section{Presentación} 
\frame{\frametitle{Presentación} 
\begin{itemize}
    \item Materia :  \emph{Teoría de Lenguajes}
    \vspace{2em}
    \item Práctica : \emph{???}
\end{itemize}
}

\section{Prácticas de la materia} 
\frame{\frametitle{Presentación} 
\begin{itemize}
    \item Primera parte
    \begin{itemize}
        \item{???}
    \end{itemize}
    \item Segunda Parte
    \begin{itemize}
        \item{???}
    \end{itemize}
\end{itemize}
}

\section{Ejercicio} 
\frame{\frametitle{Contexto}
\begin{itemize}
  \item ???
\end{itemize} 
}

\frame{\frametitle{Enunciado}

Dado $\ele = \{0^i 1^j\ |\ i > j \, \lor \, i \text{ es par}\}$

\begin{enumerate}[a.]
\item Demostrar que $\ele$ cumple
  \[\forall \alpha( \alpha \in \ele \land |\alpha| \geq 2 \implies \]
  \[
            \exists x \exists y  \exists z (
                        \alpha = x y z \land
                        |x y| \leq 2 \land
                        |y| \geq 1 \land
                        \forall i (x y^i z \in \ele))) \]
\item Demostrar que $\ele$ no es regular.
\end{enumerate}
}

\frame{
\huge
\begin{center}
¿Preguntas?
\end{center}
}

\end{document}
      

