\documentclass[hidelinks,a4paper,11pt, nofootinbib]{article}
\usepackage[width=15.5cm, left=3cm, top=2.5cm, right=2cm, left=2cm, height= 24.5cm]{geometry}
\usepackage[spanish]{babel}
\usepackage[utf8]{inputenc}
\usepackage[T1]{fontenc}
\usepackage{amsmath}
\usepackage{amsfonts}
\usepackage{amssymb}
\usepackage[colorlinks=true, pagebackref=true]{hyperref}
\usepackage{xcolor}
\usepackage{enumerate}
\usepackage{mathpazo}

\setlength{\parindent}{4em}
\setlength{\parskip}{0.3em}

\title{Prueba de Oposición - Área Teoría}  
\author{Gonzalo Ciruelos}
\date{2 de octubre de 2017}

\newcommand{\tx}[1]{\text{#1}}
\newcommand{\secu}{\ensuremath\texttt{secu}(\alpha)}
\newcommand{\rev}{\text{Reverso}}
\newcommand{\cons}{\bullet}
\newcommand{\snoc}{\circ}
\newcommand{\nil}{\text{nil}}
\newcommand{\prim}{\text{prim}}
\newcommand{\fin}{\text{fin}}
\newcommand{\ele}{\mathcal{L}}
\newcommand{\elecomp}{\overline{\ele}}

\newcommand{\por}[1]{\overset{\text{#1}}{\equiv}}
\newcommand{\ph}{\phantom{rev1}}

\begin{document}
\maketitle

\section*{Ejercicio}

Dado $\ele = \{0^i 1^j\ |\ i > j \, \lor \, i \text{ es par}\}$

\begin{enumerate}[a.]
\item Demostrar que $\ele$ cumple
  \[\forall \alpha( \alpha \in \ele \land |\alpha| \geq 2 \implies 
            \exists x \exists y  \exists z (
                        \alpha = x y z \land
                        |x y| \leq 2 \land
                        |y| \geq 1 \land
                        \forall i (x y^i z \in \ele))) \]
\item Demostrar que $\ele$ no es regular.
\end{enumerate}
                         
\section*{Solución}

\begin{enumerate}[a.]
\item Sea $\alpha \in \ele$ tal que $|\alpha| \geq 2$.
Como $\alpha \in \ele$, entonces $\alpha = 0^i1^j$ para algunos $i$, $j$.
Nos gustaría separar por casos $i$ par o impar, pero existe un caso molesto que es el caso $i = 0$.

\begin{enumerate}
\item[\underline{$i = 0$}] Entonces $\alpha = 1^j$ con $j \geq 2$,
                           luego tomo $x = \lambda$, $y = 1$, $z = 1^{j-1}$.
    Notemos que
    \begin{itemize}
    \item $|x y| =1 \leq 2$
    \item $|y| = 1 \geq 1$
    \item $xy^rz = 1^r1^{j-1} = 1^{r+j-1} \in \ele$, pues $i = 0$ es par.
    \end{itemize}
\item[\underline{$i > 0$}] Podemos considerar dos casos para $\alpha = 0^i1^j$:
  \begin{itemize}
  \item[\textbf{$i$ es par}] Como $i > 0$, entonces $i$ es como mínimo 2. Luego podemos tomar
               $x = \lambda$, $y = 0^2$, $z = 0^{i-2}1^j$. Notemos que
        \begin{itemize}
        \item $|x y| = 2 \leq 2$
        \item $|y| = 2 \geq 1$
        \item $xy^rz = 0^{2r}0^{i-2}1^j = 0^{2r+i-2}1^j \in \ele$, pues $2r+i-2$ es par.
        \end{itemize}
  \item[\textbf{$i$ es impar}] En este caso necesariamente $i > j$,
               pues si no $\alpha = 0^i1^j \not\in \ele$.
               En este caso basta conconsiderar $x = \lambda$, $y = 0$, $z = 0^{i-1}1^j$.
               Notemos que
        \begin{itemize}
        \item $|x y| = 1 \leq 2$
        \item $|y| = 1 \geq 1$
        \item $xy^rz = 0^{r}0^{i-1}1^j = 0^{r+i-1}1^j \in \ele$,
             pues si $r = 0$, entones $i - 1$ es par; y si $r \geq 1$ entonces $r + i - 1 \geq i > j$.
        \end{itemize}
    \end{itemize}
\end{enumerate}

\item Notemos que no podemos usar el pumping lemma directamente pues este lenguaje es un contraejemplo del recíproco.
Pero podemos utilizar la propiedad que nos dice que un lenguaje no es regular si y solo si su complemento no es regular.

Entonces intentemos probar que $\overline{\ele}$ no es regular.
Supongamos que $\overline{\ele}$ es regular y lleguemos a un absurdo.

Si fuera regular, por el pumping lemma,
  existe $n$ tal que si $\alpha \in \elecomp$ y $|\alpha| \geq n$,
  entonces $\alpha = xyz$ con $|y| \geq 1$, $|x y| \leq n$, y $xy^rz \in \elecomp$, $\forall r \geq 0$.

Consideremos $0^{2n+1}1^{2n+2}$. Como $0^{2n+1}1^{2n+2} \in \elecomp$ y $|0^{2n+1}1^{2n+2}| \geq n$,
entonces existen $x$,$y$,$z$ tal que pasa lo de arriba.

Pero entonces necesariamente $xy = 0^{k_x + k_y}$, pues $|x y| \leq n$.
En particular, $y^{k_y}$, con $k_y \geq 1$. Además, $z = 0^{2n+1-k_x-k_y}1^{2n+2}$.
Pero notar que si tomamos $r = 3$, entonces $xy^3z$ debería pertenecer a $\elecomp$. Veamos que no.

$xy^3z = 0^{k_x} 0^{3 k_y} 0^{2n+1-k_x-k_y}1^{2n+2} = 0^{2n+1+2k_x} 1^{2n+2}$.
Pero $2n + 1 + 2k_x \geq 2n + 3 > 2n$, luego $xy^3z \in \ele$, y entonces $xy^3z \not\in \elecomp$.
Absurdo, $\elecomp$ no puede ser un lenguaje regular, por lo tanto $\ele$ tampoco.


\end{enumerate}

\end{document}

